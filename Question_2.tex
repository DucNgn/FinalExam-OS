\newpage
\section{Question 2}

\subsection{Problem:}
Given the following code, where id indicates a process number (or ID), launch() indicates concurrent start
of the processes passed on its parameters, and N is an integer representing the number of processes that
will be launched (i.e. processes that will run on that system).

\begin{lstlisting}
boolean blocked[N];
for(int i = 0; i < N; i++)
 blocked[i] = false.
int turn = 0;
int other;
P(int id)
{
while(true)
{
 blocked[id] = true;
 while(turn != id)
 {
if(id == 0 || id == 2)
 other = 1;
else
 other = 0;
 while(blocked[other]) { turn = other;}
turn = id;
 }
 /* <<<< CRITICAL SECTION IS EXECUTED HERE>>>>>> */
 blocked[id] = false;
}
}
\end{lstlisting}

\subsection{launch(P(0), P(1))}
\subsubsection{Mutual exclusion}
The program achieves mutual exclusion since there is no possible outcome in which more than 1 thread enters the Critical Session at once. 

\subsubsection{Good progress}
The program achieves good progress since it manages to continue execution on time without any delay.

\subsubsection{Deadlock-free}
The program is deadlock-free since there is no possible outcome in which deadlock happens and fails the solution.

\newpage
\subsection{launch(P(0), P(1), P(2))}
\subsubsection{Mutual exclusion}
The program will not achieve \textbf{mutual exclusion}. One scenario is as following:

\begin{lstlisting}
 //.. Running P(0)
P(int id)
{
    while(true)
    {
         blocked[id] = true;
         while(turn != id)
         {
            // .. By pass the while loop because turn = id
         }
         /* > > > > > > Interruption happens < < < < < < */
         /* <<<< CRITICAL SECTION IS EXECUTED HERE>>>>>> */
         blocked[id] = false;
    }
}
\end{lstlisting}

The current global variables are as following:
\begin{lstlisting}
/* global values at the time */
blocked[0] = true
blocked[1] = false
blocked[2] = false
turn = 0
other = null
\end{lstlisting}

The process P(2) starts:
\begin{lstlisting}
// Continue from the interruption of P(0)
//.. Running P(2)
P(int id) {
    while(true) {
         blocked[id] = true; // blocked[2] = true
         while(turn != id)
         {
            if(id == 0 || id == 2)
             other = 1;      // other = 1
            else
             other = 0;      
             while(blocked[other]) { turn = other;}  // By pass the while loop as blocked[1] = false
            turn = id;      // turn = 2
     }
     /* EXECUTE CRITICAL SESSION AT THE SAME TIME AS P(0) */
     /* <<<< CRITICAL SECTION IS EXECUTED HERE>>>>>> */
     blocked[id] = false;
    }
}
\end{lstlisting}

\textbf{Explaination:} \\
As the scenario can happen, P(1) and P(2) might get into the critical session at the same time; thus, the code is not mutual exclusion.

\subsubsection{Good progress}
If there is no deadlock, the program would achieve good progress since it continues to run without delays

\newpage
\subsubsection{Deadlock-free}
Consider the scenario:

\begin{lstlisting}
 //.. Running P(0)
P(int id)
{
    while(true)
    {
         blocked[id] = true;
         while(turn != id)
         {
            // .. By pass the while loop because turn = id
         }
         /* <<<< CRITICAL SECTION IS EXECUTED HERE>>>>>> */
         /* > > > > > > Interrupted by P(1) < < < < < < */
         blocked[id] = false;
    }
}
\end{lstlisting}

The current global variables are as following:
\begin{lstlisting}
/* global values at the time */
blocked[0] = true
blocked[1] = false
blocked[2] = false
turn = 0
other = null
\end{lstlisting}

P(1) starts:
\begin{lstlisting}
// Continue from the interruption of P(0)
//.. Running P(1)
P(int id) {
    while(true) {
         blocked[id] = true; // blocked[1] = true
         while(turn != id)
         {
            if(id == 0 || id == 2)
             other = 1;      
            else
             other = 0;      // other = 0
            /* > > > > > > Interrupted by P(2) < < < < < < */
             while(blocked[other]) { turn = other;}  
            turn = id;      
     }
     /* <<<< CRITICAL SECTION IS EXECUTED HERE>>>>>> */
     blocked[id] = false;
    }
}
\end{lstlisting}


The current global variables are as following:
\begin{lstlisting}
/* global values at the time */
blocked[0] = true
blocked[1] = true
blocked[2] = false
turn = 0
other = 0
\end{lstlisting}

P(2) starts:
\begin{lstlisting}
// Continue from the interruption of P(1)
//.. Running P(2)
P(int id) {
    while(true) {
         blocked[id] = true; // blocked[2] = true
         while(turn != id)
         {
            if(id == 0 || id == 2)
             other = 1;      // other = 1
            else
             other = 0;      
            /* > > > > > > Interrupted by P(1) < < < < < < */
             while(blocked[other]) { turn = other;}  
            turn = id;      
     }
     /* <<<< CRITICAL SECTION IS EXECUTED HERE>>>>>> */
     blocked[id] = false;
    }
}
\end{lstlisting}

The current global variables are as following:
\begin{lstlisting}
/* global values at the time */
blocked[0] = true
blocked[1] = true
blocked[2] = true
turn = 0
other = 1
\end{lstlisting}

P(1) resumes:
\begin{lstlisting}
// Resume from the interruption in P(2)
//.. Running P(1)
P(int id) {
    while(true) {
         blocked[id] = true; 
         while(turn != id)
         {
         // .. .. ..
            /* > > > > > > Resume from the last in terruption < < < < < < */
             while(blocked[other]) { turn = other;}   // turn = 1  < - - - DEAD-LOCK
            turn = id;      
     }
     /* <<<< CRITICAL SECTION IS EXECUTED HERE>>>>>> */
     blocked[id] = false;
    }
}
\end{lstlisting}

The current global variables are as following:
\begin{lstlisting}
/* global values at the time */
blocked[0] = true
blocked[1] = true
blocked[2] = true
turn = 1
other = 1
\end{lstlisting}

\textbf{Explaination} \\
As a matter of fact, the while loop in P(1) will start the dead lock. 
\begin{itemize}
    \item If P(2) resumes from its last interruption, it will run the same while-loop as P(1).
    \item If P(0) resumes from its last interruption, it will execute the critical session and then get to the same while-loop as P(1) has
\end{itemize}
This scenario produces a possible dead lock situation of the code. Therefore, the code does not achive deadlock-free if \textbf{launch} is called as: \textbf{launch(P(0), P(1), P(2))}.
