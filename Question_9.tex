\newpage
\section{Question 9}

\subsection{Problem:}
Consider a demand-paged system that has a fast associative memory (TLB, or cache). To speed up access
to pages, it is desired to place the page table in the cache. However, since the page table of a process is
too large, the system included only the first 20\% of the page table in the cache, 30\% in the main memory,
and the rest of the table had to be kept in the secondary memory (i.e. page faults may be needed for the
page table itself!). Consider uniform request for each of the pages (i.e. each page is requested more or less
the same amount of times). To simplify the calculations, assume that cache reference costs 30 ms
(milliseconds), memory access costs 200 ms, and page fault time costs 900 ms if the fault is for the page
table, and 2 seconds if the fault is for a page. The page fault time for a page includes exactly the following:
obtaining the page from the secondary memory, possible swapping, and updating the page table. 

\subsection{Average effective access time for a page}
\textbf{Problem:} \\
What is the average effective access time for a page, if that page is in the main memory? \\
Explain clearly; you must show all your calculations; and provide the exact final answer. \\

\textbf{Answer:} \\

\begin{itemize}
    \item If cache hits, then access cache to look up the TLB 
    \item If cache misses, then try accessing main memory. 
    \item If main memory hits, then access main memory and look up the TLB
    \item If main memory misses $\implies$ it is a page fault for the page table
    \item $\alpha$ : Cache hit ratio
    \item $\beta$ : Main memory hit ratio

\end{itemize}

\begin{align*}
    \text{TLB SEARCH TIME} =  (\alpha)*(\text{Cache access time}) + \\
    ( 1 - \alpha )(\beta * \text{Memory access time} + (1-\beta)*\text{page fault}) \\
    = (0.2)(30) + (1 - 0.2)[0.3 * 200 + 0.7 * 900] \\
    = 558 ms (milliseconds) \\
    \implies \text{Average EAT} = \text{TLB SEARCH TIME} + \text{Memory access time} \\
    = 558 + 200 \\
    = 758 ms.
\end{align*}

\subsection{Average effective access time for a page if page is NOT in main memory}
What is the average effective access time for a page, if that page is NOT in the main memory? \\
Explain clearly; you must show all your calculations; and provide the exact final answer. \\

Since the page is not in main memory, it leads to a page fault for a page. \\
\textit{Note:} 2 seconds = 2000ms
\begin{align*}
    \implies \text{Average EAT} = \text{TLB SEARCH TIME} + \text{Page fault time} \\
    = 558 + 2000 \\
    = 2558 ms.
\end{align*}
